%%%%%%%%%%%%%%%%%%%%%%%%%%%%%%%%%%%%%%%%%%%%%%%%%%%%%%%%%%%%%%%%%%%%%%%%%%%%%%%%
%2345678901234567890123456789012345678901234567890123456789012345678901234567890
%        1         2         3         4         5         6         7         8

\documentclass[letterpaper, 10 pt, conference]{ieeeconf}  % Comment this line out if you need a4paper

%\documentclass[a4paper, 10pt, conference]{ieeeconf}      % Use this line for a4 paper

\IEEEoverridecommandlockouts                              % This command is only needed if 
                                                          % you want to use the \thanks command

\overrideIEEEmargins                                      % Needed to meet printer requirements.

% See the \addtolength command later in the file to balance the column lengths
% on the last page of the document

% The following packages can be found on http:\\www.ctan.org
%\usepackage{graphics} % for pdf, bitmapped graphics files
%\usepackage{epsfig} % for postscript graphics files
%\usepackage{mathptmx} % assumes new font selection scheme installed
%\usepackage{times} % assumes new font selection scheme installed
%\usepackage{amsmath} % assumes amsmath package installed
%\usepackage{amssymb}  % assumes amsmath package installed

\title{\LARGE \bf
Influence of child-robot spacial arrangement in a learning by teaching task}


\author{A1$^{1}$ and A2$^{2}$% <-this % stops a space
\thanks{}% <-this % stops a space
\thanks{$^{1}$A1 is with Faculty of A1,
        University of A1, country
        {\tt\small A1 email}}%
\thanks{$^{2}$A2 is with Faculty of A2,
        University of A2, country
        {\tt\small A2 email}}%
}


\begin{document}



\maketitle
\thispagestyle{empty}
\pagestyle{empty}


%%%%%%%%%%%%%%%%%%%%%%%%%%%%%%%%%%%%%%%%%%%%%%%%%%%%%%%%%%%%%%%%%%%%%%%%%%%%%%%%
\begin{abstract}
In this paper we present a study in which we test the influence of child-robot spatial arrangement on child's focus of attention, child's perception of robot's performance in the CoWriter learning by teaching activity.
In this activity the child teaches a Nao obot how to handwrite. 
In our study, we explore two spatial condition from Kendon's F formation, the side-by-side and the face-t-face formations. 

We have 3 conditions following Kendon's F formation: 
\begin{itemize}
\item side by side
\item face to face
\item in L shape
\end{itemize}



\end{abstract}


%%%%%%%%%%%%%%%%%%%%%%%%%%%%%%%%%%%%%%%%%%%%%%%%%%%%%%%%%%%%%%%%%%%%%%%%%%%%%%%%
\section{INTRODUCTION}
Cowriter

In this activity the child plays the role of a teacher. 



\section{RELATED WORKS}
Proxemics

Role attribution

Kendon F Formation in HRI
\cite{huttenrauch2006investigating}

We propose to study the impact of spatial arrangement on engagement of children a handwriting task
\section{APPROACH}

within subject, 2 robots in front of the child who has to teach to the two robots


Hypothesis
\begin{itemize}
\item Child feels more in a teacher student relationship when facing the robot
\item child feels more in a peer to peer when the robot is in L or side by side 
\item how does it influence the engagement -> with-me-ness
\item  does the child looks more at the experimenter when the robot is badly behaving
\item how does the child rates the robots performances according to  the arrangement
\end{itemize}


\section{METHOD}

\section{RESULTS}

\subsection{Seriousness}

\subsection{With-me-ness}
Present it as a measure of synchrony, 

Get time when robot looking where and see if similar pattern with child

\subsection{Performances}

\subsection{Questionnaire on likeability and Interrelationship}

\section{CONCLUSIONS}



\addtolength{\textheight}{-12cm}   % This command serves to balance the column lengths
                                  % on the last page of the document manually. It shortens
                                  % the textheight of the last page by a suitable amount.
                                  % This command does not take effect until the next page
                                  % so it should come on the page before the last. Make
                                  % sure that you do not shorten the textheight too much.

%%%%%%%%%%%%%%%%%%%%%%%%%%%%%%%%%%%%%%%%%%%%%%%%%%%%%%%%%%%%%%%%%%%%%%%%%%%%%%%%



%%%%%%%%%%%%%%%%%%%%%%%%%%%%%%%%%%%%%%%%%%%%%%%%%%%%%%%%%%%%%%%%%%%%%%%%%%%%%%%%



%%%%%%%%%%%%%%%%%%%%%%%%%%%%%%%%%%%%%%%%%%%%%%%%%%%%%%%%%%%%%%%%%%%%%%%%%%%%%%%%

\section*{ACKNOWLEDGMENT}




%%%%%%%%%%%%%%%%%%%%%%%%%%%%%%%%%%%%%%%%%%%%%%%%%%%%%%%%%%%%%%%%%%%%%%%%%%%%%%%%




\bibliographystyle{abbrv}
\bibliography{fformation_cowriter}


\end{document}
